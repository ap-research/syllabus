\documentclass[]{article}
\usepackage{lmodern}
\usepackage{amssymb,amsmath}
\usepackage{ifxetex,ifluatex}
\usepackage{fixltx2e} % provides \textsubscript
\ifnum 0\ifxetex 1\fi\ifluatex 1\fi=0 % if pdftex
  \usepackage[T1]{fontenc}
  \usepackage[utf8]{inputenc}
\else % if luatex or xelatex
  \ifxetex
    \usepackage{mathspec}
  \else
    \usepackage{fontspec}
  \fi
  \defaultfontfeatures{Ligatures=TeX,Scale=MatchLowercase}
\fi
% use upquote if available, for straight quotes in verbatim environments
\IfFileExists{upquote.sty}{\usepackage{upquote}}{}
% use microtype if available
\IfFileExists{microtype.sty}{%
\usepackage{microtype}
\UseMicrotypeSet[protrusion]{basicmath} % disable protrusion for tt fonts
}{}
\usepackage[margin=1in]{geometry}
\usepackage{hyperref}
\hypersetup{unicode=true,
            pdftitle={AP Research Syllabus},
            pdfborder={0 0 0},
            breaklinks=true}
\urlstyle{same}  % don't use monospace font for urls
\usepackage{natbib}
\bibliographystyle{apalike}
\usepackage{longtable,booktabs}
\usepackage{graphicx,grffile}
\makeatletter
\def\maxwidth{\ifdim\Gin@nat@width>\linewidth\linewidth\else\Gin@nat@width\fi}
\def\maxheight{\ifdim\Gin@nat@height>\textheight\textheight\else\Gin@nat@height\fi}
\makeatother
% Scale images if necessary, so that they will not overflow the page
% margins by default, and it is still possible to overwrite the defaults
% using explicit options in \includegraphics[width, height, ...]{}
\setkeys{Gin}{width=\maxwidth,height=\maxheight,keepaspectratio}
\IfFileExists{parskip.sty}{%
\usepackage{parskip}
}{% else
\setlength{\parindent}{0pt}
\setlength{\parskip}{6pt plus 2pt minus 1pt}
}
\setlength{\emergencystretch}{3em}  % prevent overfull lines
\providecommand{\tightlist}{%
  \setlength{\itemsep}{0pt}\setlength{\parskip}{0pt}}
\setcounter{secnumdepth}{5}
% Redefines (sub)paragraphs to behave more like sections
\ifx\paragraph\undefined\else
\let\oldparagraph\paragraph
\renewcommand{\paragraph}[1]{\oldparagraph{#1}\mbox{}}
\fi
\ifx\subparagraph\undefined\else
\let\oldsubparagraph\subparagraph
\renewcommand{\subparagraph}[1]{\oldsubparagraph{#1}\mbox{}}
\fi

%%% Use protect on footnotes to avoid problems with footnotes in titles
\let\rmarkdownfootnote\footnote%
\def\footnote{\protect\rmarkdownfootnote}

%%% Change title format to be more compact
\usepackage{titling}

% Create subtitle command for use in maketitle
\providecommand{\subtitle}[1]{
  \posttitle{
    \begin{center}\large#1\end{center}
    }
}

\setlength{\droptitle}{-2em}

  \title{AP Research Syllabus}
    \pretitle{\vspace{\droptitle}\centering\huge}
  \posttitle{\par}
    \author{}
    \preauthor{}\postauthor{}
      \predate{\centering\large\emph}
  \postdate{\par}
    \date{2019-06-02}

\usepackage{booktabs}
\usepackage{amsthm}
\makeatletter
\def\thm@space@setup{%
  \thm@preskip=8pt plus 2pt minus 4pt
  \thm@postskip=\thm@preskip
}
\makeatother

%Allow for more levels in bullet lists
%https://github.com/Witiko/markdown/issues/2
%https://tex.stackexchange.com/questions/41408/a-five-level-deep-list
\usepackage{enumitem}
\setlistdepth{20}
\renewlist{itemize}{itemize}{20}
% initially, use dots for all levels
\setlist[itemize]{label=$\cdot$}

% customize the first 3 levels
\setlist[itemize,1]{label=\textbullet}
\setlist[itemize,2]{label=--}
\setlist[itemize,3]{label=*}

%https://bookdown.org/yihui/bookdown/latex-index.html
\usepackage{makeidx}
\makeindex

\begin{document}
\maketitle

{
\setcounter{tocdepth}{2}
\tableofcontents
}
\hypertarget{course-description}{%
\section{Course Description}\label{course-description}}

\textbf{Prerequisite:} AP Seminar\\
AP Research is a one-year course that culminates in a 4000- to 5000-word academic paper and a 15- to 20-minute presentation with oral defense. Students will learn the process of academic research as well as industry-aligned research tools. Depending on the students' research topics, research methods aligned with the AP Statistics curriculum may be integrated into mini-workshops.

\hypertarget{course-objectives}{%
\section{Course Objectives}\label{course-objectives}}

\begin{enumerate}
\def\labelenumi{\arabic{enumi}.}
\tightlist
\item
  Develop research proposal that demonstrates ability to define research topic and question with accompanying literature review.
\item
  Design proper research procedures that apply sound methodology to address research question.
\item
  Analyze and evaluate research findings with appropriate methods.
\item
  Apply general research tools such as source management software and version control to organize academic paper.
\item
  Gain familiarity with academic and industry standards of reproducible research.
\item
  Communicate research effectively with well crafted presentation slides.
\end{enumerate}

\hypertarget{recommended-resources}{%
\section{Recommended Resources}\label{recommended-resources}}

Creswell, J. W. (2009). \emph{Research Design: Qualitative,
Quantitative, and Mixed Methods Approaches}. Thousand Oaks, CA:
SAGE Publications.

Gray, P. S, J. B. Williamson, D. A. Karp, and J. R. Dalphin
(2007). \emph{The Research Imagination: An Introducation to Qualitative
and Quantitative Methods}. New York, NY: Cambridge University
Press.

\hypertarget{curricular-requirements}{%
\section{Curricular Requirements}\label{curricular-requirements}}

The \href{https://secure-media.collegeboard.org/digitalServices/pdf/ap/ap-course-audit/ap-research-syllabus-development-guide.pdf}{College Board Syllabus Development Guide} lists the following curricular requirements for AP Research:

\begin{quote}
\textbf{CR1a:} Students develop and apply discrete skills identified in the learning objectives within the Big Idea 1: Question and Explore.
\end{quote}

\begin{quote}
\textbf{CR1b:} Students develop and apply discrete skills identified in the learning objectives within the Big Idea 2: Understand and Analyze.
\end{quote}

\begin{quote}
\textbf{CR1c:} Students develop and apply discrete skills identified in the learning objectives within the Big Idea 3: Evaluate Multiple Perspectives.
\end{quote}

\begin{quote}
\textbf{CR1d:} Students develop and apply discrete skills identified in the learning objectives within the Big Idea 4: Synthesize Ideas.
\end{quote}

\begin{quote}
\textbf{CR1e:} Students develop and apply collaboration skills identified in the learning objectives within the Big Idea 5: Team, Transform, and Transmit.
\end{quote}

\begin{quote}
\textbf{CR1f:} Students develop and apply reflection skills identified in the learning objectives within the Big Idea 5: Team, Transform, and Transmit.
\end{quote}

\begin{quote}
\textbf{CR1g:} Students develop and apply written and oral communication skills identified in the learning objectives within the Big Idea 5: Team, Transform, and Transmit.
\end{quote}

\begin{quote}
\textbf{CR2a:} Students develop an understanding of ethical research practices.
\end{quote}

\begin{quote}
\textbf{CR2b:} Students develop an understanding of the AP Capstone Policy on Plagiarism and Falsification or Fabrication of Information.
\end{quote}

\begin{quote}
\textbf{CR3:} In the classroom and independently (while possibly consulting any expert advisors), students learn and employ research and inquiry methods to develop, manage, and conduct an in-depth investigation of an area of personal interest, culminating in an academic paper of 4,000-5,000 words that includes the following elements:

\begin{itemize}
\tightlist
\item
  Introduction
\item
  Method, Process, or Approach
\item
  Results, Product, or Findings
\item
  Discussion, Analysis, and/or Evaluation
\item
  Conclusion and Future Directions
\item
  Bibliography
\end{itemize}
\end{quote}

\begin{quote}
\textbf{CR4a:} Students document their inquiry processes, communicate with their teachers and any expert advisors, and reflect on their thought processes.
\end{quote}

\begin{quote}
\textbf{CR4b:} Students have regular work-in-progress interviews with their teachers to review their progress and to receive feedback on their scholarly work as evidenced by the PREP.
\end{quote}

\begin{quote}
\textbf{CR5:} Students develop and deliver a presentation (using an appropriate medium) and an oral defense to a panel on their research processes, method, and findings.
\end{quote}

Testing testing. \index{CR5}

\hypertarget{course-schedule}{%
\section{Course Schedule}\label{course-schedule}}

\hypertarget{unit-1}{%
\subsection{Unit 1}\label{unit-1}}

\hypertarget{unit-2}{%
\subsection{Unit 2}\label{unit-2}}

\hypertarget{unit-3}{%
\subsection{Unit 3}\label{unit-3}}

\hypertarget{unit-4}{%
\subsection{Unit 4}\label{unit-4}}

\bibliography{book.bib,packages.bib}

\pdfbookmark[0]{\indexname}{Index}
\printindex


\end{document}
