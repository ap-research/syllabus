\documentclass[11pt,]{article}
\usepackage{lmodern}
\usepackage{amssymb,amsmath}
\usepackage{ifxetex,ifluatex}
\usepackage{fixltx2e} % provides \textsubscript
\ifnum 0\ifxetex 1\fi\ifluatex 1\fi=0 % if pdftex
  \usepackage[T1]{fontenc}
  \usepackage[utf8]{inputenc}
\else % if luatex or xelatex
  \ifxetex
    \usepackage{mathspec}
  \else
    \usepackage{fontspec}
  \fi
  \defaultfontfeatures{Ligatures=TeX,Scale=MatchLowercase}
\fi
% use upquote if available, for straight quotes in verbatim environments
\IfFileExists{upquote.sty}{\usepackage{upquote}}{}
% use microtype if available
\IfFileExists{microtype.sty}{%
\usepackage{microtype}
\UseMicrotypeSet[protrusion]{basicmath} % disable protrusion for tt fonts
}{}
\usepackage[margin=1in]{geometry}
\usepackage{hyperref}
\hypersetup{unicode=true,
            pdftitle={AP® Research Syllabus},
            pdfborder={0 0 0},
            breaklinks=true}
\urlstyle{same}  % don't use monospace font for urls
\usepackage{longtable,booktabs}
\usepackage{graphicx,grffile}
\makeatletter
\def\maxwidth{\ifdim\Gin@nat@width>\linewidth\linewidth\else\Gin@nat@width\fi}
\def\maxheight{\ifdim\Gin@nat@height>\textheight\textheight\else\Gin@nat@height\fi}
\makeatother
% Scale images if necessary, so that they will not overflow the page
% margins by default, and it is still possible to overwrite the defaults
% using explicit options in \includegraphics[width, height, ...]{}
\setkeys{Gin}{width=\maxwidth,height=\maxheight,keepaspectratio}
\IfFileExists{parskip.sty}{%
\usepackage{parskip}
}{% else
\setlength{\parindent}{0pt}
\setlength{\parskip}{6pt plus 2pt minus 1pt}
}
\setlength{\emergencystretch}{3em}  % prevent overfull lines
\providecommand{\tightlist}{%
  \setlength{\itemsep}{0pt}\setlength{\parskip}{0pt}}
\setcounter{secnumdepth}{5}
% Redefines (sub)paragraphs to behave more like sections
\ifx\paragraph\undefined\else
\let\oldparagraph\paragraph
\renewcommand{\paragraph}[1]{\oldparagraph{#1}\mbox{}}
\fi
\ifx\subparagraph\undefined\else
\let\oldsubparagraph\subparagraph
\renewcommand{\subparagraph}[1]{\oldsubparagraph{#1}\mbox{}}
\fi

%%% Use protect on footnotes to avoid problems with footnotes in titles
\let\rmarkdownfootnote\footnote%
\def\footnote{\protect\rmarkdownfootnote}

%%% Change title format to be more compact
\usepackage{titling}

% Create subtitle command for use in maketitle
\providecommand{\subtitle}[1]{
  \posttitle{
    \begin{center}\large#1\end{center}
    }
}

\setlength{\droptitle}{-2em}

  \title{AP\textsuperscript{®} Research Syllabus}
    \pretitle{\vspace{\droptitle}\centering\huge}
  \posttitle{\par}
    \author{}
    \preauthor{}\postauthor{}
      \predate{\centering\large\emph}
  \postdate{\par}
    \date{\emph{2019-07-25}}

% \usepackage[letterpaper, margin=2in]{geometry}
\usepackage{booktabs}
\usepackage{amsthm}
\makeatletter
\def\thm@space@setup{%
  \thm@preskip=8pt plus 2pt minus 4pt
  \thm@postskip=\thm@preskip
}
\makeatother

%Allow for more levels in bullet lists
%https://github.com/Witiko/markdown/issues/2
%https://tex.stackexchange.com/questions/41408/a-five-level-deep-list
\usepackage{enumitem}
\setlistdepth{20}
\renewlist{itemize}{itemize}{20}
% initially, use dots for all levels
\setlist[itemize]{label=$\cdot$}

% customize the first 3 levels
\setlist[itemize,1]{label=\textbullet}
\setlist[itemize,2]{label=--}
\setlist[itemize,3]{label=*}

\usepackage{xcolor}
\usepackage{hyperref}
\hypersetup{
    colorlinks = true,
    linkbordercolor = {white},
    linkcolor = {black},
    urlcolor = {blue},
    citecolor = {blue},
    anchorcolor = {blue},
    bookmarksopen = true,
    bookmarksdepth=2
}

\usepackage{bibentry}
\usepackage{hanging}
%https://bookdown.org/yihui/bookdown/latex-index.html
%https://tex.stackexchange.com/questions/57427/how-to-add-printindex-to-tableofcontents
\usepackage{makeidx}
% \usepackage{imakeidx}
% \indexsetup{level=\section}
% \usepackage[totoc]{idxlayout}
\makeindex
\usepackage[nottoc]{tocbibind}

% \setlength{\bibhang}{0.5in}

\begin{document}
\maketitle

{
\setcounter{tocdepth}{2}
\tableofcontents
}
\newpage
\hypersetup{linkcolor=blue}

\hypertarget{course-description}{%
\section{Course Description}\label{course-description}}

\textbf{Prerequisite:} AP\textsuperscript{®} Seminar\\
AP\textsuperscript{®} Research is a one-year course that culminates in a 4000- to 5000-word academic paper and a 20-minute presentation with oral defense. Students will learn the process of academic research as well as industry-aligned research tools.

\hypertarget{course-objectives}{%
\section{Course Objectives}\label{course-objectives}}

\begin{enumerate}
\def\labelenumi{\arabic{enumi}.}
\tightlist
\item
  Develop research proposal that demonstrates ability to define research topic and question with accompanying literature review.
\item
  Design proper research procedures that apply sound methods to address research question.
\item
  Analyze and evaluate research findings with appropriate methods.
\item
  Apply general research tools such as source management software and version control to organize academic paper.
\item
  Gain familiarity with academic and industry standards of reproducible research.
\item
  Communicate research effectively with well crafted presentation slides.
\end{enumerate}

\hypertarget{policies}{%
\section{Policies}\label{policies}}

\hypertarget{rules-and-expectations}{%
\subsection{Rules and Expectations}\label{rules-and-expectations}}

\begin{enumerate}
\def\labelenumi{\arabic{enumi}.}
\item
  Respect

  \begin{itemize}
  \tightlist
  \item
    Please observe and uphold the Golden Rule by treating others as you would like to be treated.
  \item
    We should all be respectful of each other in all written and verbal communications, especially when providing constructive feedback during the peer review process.
  \end{itemize}
\item
  Responsibility

  \begin{itemize}
  \tightlist
  \item
    Take ownership of your learning and research process. You are the expert of your chosen topic.
  \item
    We all have a shared responsibility to help each other grow through the peer review process.
  \end{itemize}
\item
  Resilience

  \begin{itemize}
  \tightlist
  \item
    Research is a long and arduous process, often with unforeseen obstacles and undefined paths.
  \item
    You should persevere even in the face of adversity. Don't let setbacks discourage you from moving ahead with your research.
  \end{itemize}
\end{enumerate}

\hypertarget{ap-capstone-policy-on-plagiarism-and-falsification-or-fabrication-of-information}{%
\subsection{AP Capstone Policy on Plagiarism and Falsification or Fabrication of Information}\label{ap-capstone-policy-on-plagiarism-and-falsification-or-fabrication-of-information}}

Please read and comply with the following policy issued by College Board (\protect\hyperlink{ref-cb}{2019}):

\begin{quote}
Participating teachers shall inform students of the consequences of plagiarism and instruct students to ethically use and acknowledge the ideas and work of others throughout their coursework. The student's individual voice should be clearly evident, and the ideas of others must be acknowledged, attributed, and/or cited.
\end{quote}

\begin{quote}
A student who fails to acknowledge the source or author of any and all information or evidence taken from the work of someone else through cita
tion, attribution or reference in the body of the work, or through a bibliographic entry, will receive a score of 0 on that particular component of the AP Seminar and/or AP Research Performance Task. In AP Seminar, a team of students that fails to properly acknowledge sources or authors on the Team Multimedia Presentation will receive a group score of 0 for that component of the Team Project and Presentation.A student who incorporates falsified or fabricated information (e.g.~evidence, data, sources, and/or authors) will receive a score of 0 on that particular component of the AP Seminar and/or AP Research Performance Task. In AP Seminar, a team of students that incorporates falsified or fabricated information in the Team Multimedia Presentation will receive a group score of 0 for that component of the Team Project and Presentation.
\end{quote}

\begin{quote}
A student who incorporates falsified or fabricated information (e.g., evidence, data, sources, and/or authors) will receive a score of 0 on that particular component of the AP Seminar and/or AP Research Performance Task. In AP Seminar, a team of students that incorporates falsified or fabricated information in the Team Multimedia Presentation will receive a group score of 0 for that component of the Team Project and Presentation (p.~14).
\end{quote}

\hypertarget{expert-advisors}{%
\subsection{Expert Advisors}\label{expert-advisors}}

You are allowed to seek expert advisors to provide general feedback on your research. Expert advisors may include on-campus faculty members or other experts in your field of inquiry. Please check with your AP Research teacher before contacting potential expert advisors.

For more details on policies regarding expert advisors, please refer to pages 51--52 of the \href{https://apcentral.collegeboard.org/pdf/ap-research-course-and-exam-description.pdf}{AP Research Course and Exam Description}. You teacher may inform you of additional school policies regarding communication with expert advisors.

\hypertarget{institutional-review-board-irb}{%
\subsection{Institutional Review Board (IRB)}\label{institutional-review-board-irb}}

If your research involves human subjects, you will most likely need to go through an Institutional Review Board (IRB) process to ensure compliance with research ethics. Please read page 44 of the \href{https://apcentral.collegeboard.org/pdf/ap-research-course-and-exam-description.pdf}{AP Research Course and Exam Description} for more information. You may go through the IRB approval process in two ways:

\begin{enumerate}
\def\labelenumi{\arabic{enumi}.}
\tightlist
\item
  IRB at nearby university or local science fair committee
\item
  IRB committee established at your school
\end{enumerate}

You may not collect any data or conduct any research involving human subjects until an IRB committee has approved your proposed research. Please see your teacher to discuss IRB options if your research will involve human subjects. \textbf{{[}CR2a{]}} \index{CR2a}

\hypertarget{consent-forms}{%
\subsection{Consent Forms}\label{consent-forms}}

You must receive the written consent of all research participants. Your teacher may provide you with a consent form template that you can adapt to fit your research method. The adapted consent form should be appended to the IRB form for review prior to conducting your research. If research participants are under the age of 18, their parents will also need to sign the consent form. \textbf{{[}CR2a{]}} \index{CR2a}

\hypertarget{textbooks}{%
\section{Textbooks}\label{textbooks}}

College Board (2017). \emph{AP Research 2017 Student Workbook}.

Creswell, J. W. (2009). \emph{Research Design: Qualitative,
Quantitative, and Mixed Methods Approaches}. Thousand Oaks, CA:
SAGE Publications.

Gray, P. S., Williamson, J. B., Karp, D. A., \& Dalphin, J. R.
(2007). \emph{The Research Imagination: An Introducation to Qualitative
and Quantitative Methods}. New York, NY: Cambridge University
Press.

You are not required to purchase these texts, since the relevant pages will be provided to you as reference. In this class, we will refer to the texts above as Workbook, Creswell, and Gray et al., respectively.

\hypertarget{deadlines-and-important-dates}{%
\section{Deadlines and Important Dates}\label{deadlines-and-important-dates}}

Please note that some of the following deadlines may be earlier than the official College Board deadlines as a buffer for unforeseen issues such as technical difficulties. These dates are subject to change. Your teacher will inform you via e-mail regarding any deadline adjustments.

\begin{itemize}
\tightlist
\item
  October 31, 2019: Inquiry Proposal Form
\item
  November 29, 2019: Research Proposal
\item
  December 20, 2019: Annotated Bibliography
\item
  February 28, 2020: Academic Paper (first draft)
\item
  March 16, 2020: Academic Paper (second draft)
\item
  April 1--15, 2020: Presentation and Oral Defense
\item
  April 25, 2020: Final Academic Paper
\item
  May 31, 2020: Process and Reflection Portfolio (PREP)
\end{itemize}

\hypertarget{curricular-requirements}{%
\section{Curricular Requirements}\label{curricular-requirements}}

The \href{https://secure-media.collegeboard.org/digitalServices/pdf/ap/ap-course-audit/ap-research-syllabus-development-guide.pdf}{College Board Syllabus Development Guide} lists the following curricular requirements for AP Research:

\begin{quote}
\textbf{CR1a:} Students develop and apply discrete skills identified in the learning objectives within the Big Idea 1: Question and Explore.
\end{quote}

\begin{quote}
\textbf{CR1b:} Students develop and apply discrete skills identified in the learning objectives within the Big Idea 2: Understand and Analyze.
\end{quote}

\begin{quote}
\textbf{CR1c:} Students develop and apply discrete skills identified in the learning objectives within the Big Idea 3: Evaluate Multiple Perspectives.
\end{quote}

\begin{quote}
\textbf{CR1d:} Students develop and apply discrete skills identified in the learning objectives within the Big Idea 4: Synthesize Ideas.
\end{quote}

\begin{quote}
\textbf{CR1e:} Students develop and apply collaboration skills identified in the learning objectives within the Big Idea 5: Team, Transform, and Transmit.
\end{quote}

\begin{quote}
\textbf{CR1f:} Students develop and apply reflection skills identified in the learning objectives within the Big Idea 5: Team, Transform, and Transmit.
\end{quote}

\begin{quote}
\textbf{CR1g:} Students develop and apply written and oral communication skills identified in the learning objectives within the Big Idea 5: Team, Transform, and Transmit.
\end{quote}

\begin{quote}
\textbf{CR2a:} Students develop an understanding of ethical research practices.
\end{quote}

\begin{quote}
\textbf{CR2b:} Students develop an understanding of the AP Capstone Policy on Plagiarism and Falsification or Fabrication of Information.
\end{quote}

\begin{quote}
\textbf{CR3:} In the classroom and independently (while possibly consulting any expert advisors), students learn and employ research and inquiry methods to develop, manage, and conduct an in-depth investigation of an area of personal interest, culminating in an academic paper of 4,000-5,000 words that includes the following elements:

\begin{itemize}
\tightlist
\item
  Introduction
\item
  Method, Process, or Approach
\item
  Results, Product, or Findings
\item
  Discussion, Analysis, and/or Evaluation
\item
  Conclusion and Future Directions
\item
  Bibliography
\end{itemize}
\end{quote}

\begin{quote}
\textbf{CR4a:} Students document their inquiry processes, communicate with their teachers and any expert advisors, and reflect on their thought processes.
\end{quote}

\begin{quote}
\textbf{CR4b:} Students have regular work-in-progress interviews with their teachers to review their progress and to receive feedback on their scholarly work as evidenced by the PREP.
\end{quote}

\begin{quote}
\textbf{CR5:} Students develop and deliver a presentation (using an appropriate medium) and an oral defense to a panel on their research processes, method, and findings (College Board, \protect\hyperlink{ref-syllabus}{n.d.}).
\end{quote}

\hypertarget{deliverables}{%
\section{Deliverables}\label{deliverables}}

\hypertarget{annotated-bibliography}{%
\subsection{Annotated Bibliography}\label{annotated-bibliography}}

In preparation for the literature review section of your paper, you will create an annotated bibliography of your collected sources. You will describe the key results of each source and show the connection to your research question. As you expand the annotated bibliography, you should consider how you will synthesize the sources into a cohesive literature review for your final paper. Furthermore, your search will allow you to uncover a gap in the literature that you will include in your paper to motivate the relevance and originality of your research question.

\hypertarget{inquiry-proposal-form}{%
\subsection{Inquiry Proposal Form}\label{inquiry-proposal-form}}

Before students begin conducting their research, they must submit an inquiry proposal form to their teacher for approval. Research that involves human subjects will go through an Institutional Review Board (IRB) process to ensure compliance with research ethics. If your research requires IRB review, your inquiry proposal form should include a completed IRB form as well as any accompanying documents such as proposed surveys, interview questions, or questionnaires. Your teacher will not grant approval of your inquiry proposal form until your proposed research has passed IRB review. \textbf{{[}CR2a{]} {[}CR3{]}} \index{CR2a} \index{CR3}

During the approval process of your inquiry proposal form, you are not allowed to begin conducting research. However, in the meanwhile, you should continue your literature review and study general methods applicable to your proposed research as preparation.

\hypertarget{poster-presentation}{%
\subsection{Poster Presentation}\label{poster-presentation}}

Your initial practice and formulation of your final presentation will be an informal poster presentation in which you will present a short elevator speech of your research topic with a poster that contains a problem statement, research question, hypotheses, proposed methods, and references. \textbf{{[}CR1g{]} {[}CR5{]}} \index{CR1g} \index{CR5}

\hypertarget{research-blueprint-poster}{%
\subsection{Research Blueprint Poster}\label{research-blueprint-poster}}

After you have learned more about the research design and methods appropriate for your research, you will design an informal research blueprint poster that outlines your proposed methods. This poster will help you design the methods section of your final presentation slides and provide an opportunity for peer review before you delineate your research methods in more detail in the research proposal. \textbf{{[}CR1g{]} {[}CR5{]}} \index{CR1g} \index{CR5}

\hypertarget{research-proposal}{%
\subsection{Research Proposal}\label{research-proposal}}

After your teacher approves your inquiry proposal form, you may proceed with your research and develop a lengthier research proposal to fine tune and expand upon elements of the inquiry proposal form. The research proposal will contain an introduction with a focused research question and hypothesis, initial draft of a literature review, and details of proposed methods. Many elements of the research proposal will be similar to your responses in the inquiry proposal form, but some sections such as the literature review should include more synthesis and reflect your teacher's feedback on the inquiry proposal form. This research proposal can serve as a template for the first draft of your academic paper. \textbf{{[}CR3{]}} \index{CR3}

\hypertarget{process-and-reflection-portfolio-prep}{%
\subsection{Process and Reflection Portfolio (PREP)}\label{process-and-reflection-portfolio-prep}}

In order to keep track of the research process, students will keep a digital Process and Reflection Portfolio (PREP). The final PREP version should document each step of the research process and include the following items: \textbf{{[}CR4a{]}} \index{CR4a}

\begin{itemize}
\tightlist
\item
  Readme file
\item
  Research question (formulation process)
\item
  Annotated bibliography
\item
  IRB and consent forms
\item
  Documentation of interactions with expert advisors
\item
  Inquiry proposal form
\item
  Research proposal
\item
  Paper drafts
\item
  Data with documentation
\item
  Peer review (comments/reflection)
\item
  Poster presentation
\item
  Presentation slides
\item
  Oral defense preparation questions
\item
  Reflection on research process
\item
  Student signature of attestation: ``I affirm that all work contained in this Process and Reflection Portfolio (PREP) is my own and complies with the AP Capstone Policy on Plagiarism and Falsification or Fabrication of Information.''
\end{itemize}

In your weekly check-ins with your teacher, you will use your PREP to demonstrate your research progress. \textbf{{[}CR4b{]}} \index{CR4b}

\hypertarget{academic-paper}{%
\subsection{Academic Paper}\label{academic-paper}}

Your final academic paper should contain 4000--5000 words and include the following components: \textbf{{[}CR3{]}} \index{CR3}

\begin{enumerate}
\def\labelenumi{\arabic{enumi}.}
\item
  Introduction

  \begin{itemize}
  \tightlist
  \item
    You should contextualize your focused research question in the field of inquiry by synthesizing a literature review that describes the current understanding of your research topic.
  \item
    After the literature review, you should identify a gap in the literature that demonstrates the value of your research question.
  \item
    Your research question should also include your hypothesis or set of hypotheses that you will test using a research method.
  \end{itemize}
\item
  Method, Process, or Approach

  \begin{itemize}
  \tightlist
  \item
    You should select and justify your choice of a qualitative, quantitative, or mixed methods approach to test your research question.
  \item
    Ensure that your method is replicable by describing each step of the process in detail with references to additional materials in the appendix if necessary.
  \end{itemize}
\item
  Results, Products, or Findings

  \begin{itemize}
  \tightlist
  \item
    Describe both the evidence and results in detail.
  \item
    Explain how you used the evidence in your research method to obtain your results.
  \end{itemize}
\item
  Discussion, Analysis, and/or Evaluation

  \begin{itemize}
  \tightlist
  \item
    Describe the implications of your findings. If your research did not yield expected results, explain why you believe the outcome did not match your hypothesis.
  \item
    Connect your findings to your research question and the broader field of inquiry.
  \item
    Address limitations of your study.
  \item
    Discuss directions for further studies.
  \end{itemize}
\end{enumerate}

\hypertarget{presentation-and-oral-defense}{%
\subsection{Presentation and Oral Defense}\label{presentation-and-oral-defense}}

You will have 20 minutes to present your research and answer questions from a three-member panel. Your AP Research teacher will be part of the panel and select two additional panel members. While the other panel members will provide notes and input regarding your presentation, only your AP Research teacher will conduct the actual scoring, which will constitute 25\% of your official AP Research score.

After your presentation, each member of the panel will ask one question from the list of oral defense questions found on pages 58--59 of the \href{https://apcentral.collegeboard.org/pdf/ap-research-course-and-exam-description.pdf}{AP Research Course and Exam Description}. You will receive one question from each section of the list. Since the oral defense component counts toward the 20-minute limit, you should consider completing your presentation in 16 minutes to allow for enough time for oral defense questions. \textbf{{[}CR5{]}} \index{CR5}

\hypertarget{grades}{%
\section{Grades}\label{grades}}

\hypertarget{ap-research-official-score}{%
\subsection{AP Research Official Score}\label{ap-research-official-score}}

\begin{itemize}
\tightlist
\item
  75\%: Academic paper (4000--5000 words) scored by College Board
\item
  25\%: Presentation and oral defense (20 minutes) scored by your AP Research teacher
\end{itemize}

\hypertarget{ap-research-course-grade}{%
\subsection{AP Research Course Grade}\label{ap-research-course-grade}}

Since the components that go into your official AP Research score cannot be part of your course grade, the grading for this course will mostly focus on the research process. The grading weights of each stage of the research process are as follows:

\begin{itemize}
\tightlist
\item
  20\%: Process and Reflection Portfolio (PREP)
\item
  10\%: Inquiry Proposal Form
\item
  10\%: Poster Presentation
\item
  10\%: Research Blueprint Poster
\item
  10\%: Research Proposal
\item
  10\%: Annotated Bibliography
\item
  10\%: 1st Draft of Academic Paper (completion/non-evaluative)
\item
  10\%: 2nd Draft of Academic Paper (completion/non-evaluative)
\item
  10\%: Final Draft of Academic Paper (completion/non-evaluative)
\end{itemize}

\hypertarget{pacing-guide}{%
\section{Pacing Guide}\label{pacing-guide}}

\hypertarget{unit-1-introduction}{%
\subsection{Unit 1: Introduction}\label{unit-1-introduction}}

\textbf{July/August}: Develop research ideas/topics and formulate focused research questions. \textbf{{[}CR1a{]} {[}CR1b{]}} \index{CR1a} \index{CR1b}

\begin{enumerate}
\def\labelenumi{\arabic{enumi}.}
\item
  Use Workbook (pp.~18--26) as a guide to turn a problem statement in your field of inquiry into a focused research question. \textbf{{[}CR1a{]}} \index{CR1a}
\item
  E-mail your teacher with your proposed research topic and focused research question. You may submit multiple topics/questions if you have not decided on just one. \textbf{{[}CR1g{]}} \index{CR1g}
\item
  Begin your PREP by documenting the process of formulating each iteration of your research question, which should have the following features: \textbf{{[}CR4a{]}} \index{CR4a}

  \begin{itemize}
  \tightlist
  \item
    focused: narrowing in scope
  \item
    valuable: contributes to a new understanding in the field
  \item
    feasible: replicable method that can be completed in a few months in time for the final paper deadline
  \end{itemize}
\end{enumerate}

\textbf{August/September}: Conduct preliminary research on a research topic. Begin annotated bibliography. Refine research question and begin research proposals. Gain familiarity with the academic paper rubric. \textbf{{[}CR1b{]} {[}CR1c{]} {[}CR1d{]}} \index{CR1b} \index{CR1c} \index{CR1d}

\begin{enumerate}
\def\labelenumi{\arabic{enumi}.}
\setcounter{enumi}{3}
\item
  Refer to Workbook (pp.~6--9) to explore different ways of knowing across disciplines. \textbf{{[}CR1c{]}} \index{CR1c}

  \begin{itemize}
  \tightlist
  \item
    In your PREP, reflect on how your chosen discipline engages in research using your collected sources as examples. \textbf{{[}CR1f{]} {[}CR4a{]}} \index{CR1f} \index{CR4a}
  \end{itemize}
\item
  Use Workbook (pp.~64--81) as a guide to begin your annotated bibliography. Focus on the following points:

  \begin{itemize}
  \tightlist
  \item
    Select a discipline-specific style (e.g., MLA, APA, or Chicago) used in your field of inquiry. Refer to Workbook (p.~65) and \href{https://owl.purdue.edu/owl/purdue_owl.html}{Purdue OWL} for detailed documentation on citation styles.
  \item
    Select and use a reference management software such as \href{https://www.mendeley.com}{Mendeley} to organize your sources and integrate your bibliography into Microsoft Word or LaTeX.
  \item
    Go through the process of SMARTER searches to ensure that sources are situated in your topic of inquiry from multiple perspectives, relevant to your research question, and integrated into the broader field of knowledge (Workbook, pp.~75--76). \textbf{{[}CR1c{]} {[}CR1d{]}} \index{CR1c} \index{CR1d}
  \item
    Begin your annotated bibliography with 5--10 sources. Add 5--10 sources every week to your annotated bibliography until you have enough sources to develop a literature review. Refer to Workbook (pp.~77-81) for sample annotated bibliography entries.
  \item
    When finding sources, you should use the PAARC test to assess credibility, validity, and relevance (Workbook, pp.~82--83).
  \end{itemize}
\item
  Go through previous AP Research \href{https://apcentral.collegeboard.org/courses/ap-research/exam/past-exam-questions?course=ap-research}{sample papers}. Annotate sample papers using the \href{https://apcentral.collegeboard.org/pdf/ap18-sg-research-academic-paper.pdf}{new AP Research paper rubic}.
\end{enumerate}

\hypertarget{unit-2-topic-to-proposal}{%
\subsection{Unit 2: Topic to Proposal}\label{unit-2-topic-to-proposal}}

\textbf{October/November}: Complete research proposals for approval. Synthesize annotated bibliography into literature review. \textbf{{[}CR1d{]} {[}CR3{]}} \index{CR1d} \index{CR3}

\begin{enumerate}
\def\labelenumi{\arabic{enumi}.}
\item
  As a class, we will go over important ethical pratices in research, including the following:

  \begin{itemize}
  \tightlist
  \item
    AP Capstone Policy on Plagiarism and Falsification or Fabrication of Information \textbf{{[}CR2b{]}} \index{CR2b}
  \item
    Institutional Review Board (IRB) process for research involving human subjects \textbf{{[}CR2a{]}} \index{CR2a}
  \item
    Consent forms for research participants \textbf{{[}CR2a{]}} \index{CR2a}
  \item
    Parental permission for research participants under age 18 \textbf{{[}CR2a{]}} \index{CR2a}
  \end{itemize}
\item
  Do a dry run of an inquiry method using the Health Halos Experiment (Workbook, pp.~148--153). As a class, use this topic to fill out an inquiry proposal form as a sample.
\item
  With you own research topic, complete an initial draft of the inquiry proposal form. \textbf{{[}CR3{]}} \index{CR3}
\item
  As your first presentation practice, you will use p.~77 of the Workbook as a reference to develop a brief elevator speech with an informal poster that contains the following elements: \textbf{{[}CR5{]}} \index{CR5}

  \begin{itemize}
  \tightlist
  \item
    Proposal title
  \item
    Problem statement \& research question
  \item
    Definitions, hypotheses, and importance of study
  \item
    Proposed research methods
  \item
    List of sources
  \end{itemize}
\item
  Develop slides to present elements of the inquiry proposal form for peer review. \textbf{{[}CR1e{]} {[}CR1g{]} {[}CR5{]}} \index{CR1e} \index{CR1g} \index{CR5}
\item
  Revise inquiry proposal form to reflect peer review comments. \textbf{{[}CR1e{]} {[}CR1f{]}} \index{CR1e} \index{CR1f}
\item
  Submit inquiry proposal form to teacher for approval. If applicable, you should include IRB forms and identify potential expert advisors. You may not begin conducting research until your teacher approves your inquiry proposal form. \textbf{{[}CR2a{]} {[}CR3{]}} \index{CR2a} \index{CR3}
\item
  Establish weekly PREP check-ins with your teacher. \textbf{{[}CR4b{]}} \index{CR4b}

  \begin{itemize}
  \tightlist
  \item
    Store your PREP on a cloud server and share a password-protected URL link with your teacher for weekly progress check-ups.
  \item
    Create a folder in your PREP to document reflections on peer review comments as well as feedback from your teacher and expert advisors. \textbf{{[}CR1f{]}} \index{CR1f}
  \end{itemize}
\end{enumerate}

\hypertarget{unit-3-research-methods}{%
\subsection{Unit 3: Research Methods}\label{unit-3-research-methods}}

\textbf{November/December}: Learn and implement replicable research methods to address research question. \textbf{{[}CR3{]}} \index{CR3}

\begin{enumerate}
\def\labelenumi{\arabic{enumi}.}
\item
  Review Chapter 3 of Gray et al.~(2007, pp.~33--56) for an overview on research design.
\item
  Go to the \href{https://libguides.usc.edu/writingguide/researchdesigns}{USC Libraries Research Guides}. Under ``Types of Research Design'' tab, skim through the various research designs to identify the one that most closely matches your proposed research design as well as those found in your annotated bibliography.

  \begin{itemize}
  \tightlist
  \item
    Learn more about your research design and the specific research methods you will employ to conduct your research.
  \item
    As a starting point, establish if you will use qualitative, quantitative, or mixed methods. Reference ``6. The Methodology'' tab in \href{https://libguides.usc.edu/writingguide/researchdesigns}{USC Libraries Research Guides} before you embark on more specific methods.
  \end{itemize}
\item
  Review Chapters 8, 9, and 10 in Creswell (2009, pp.~145--225) for quantitative, qualitative, and mixed methods, respectively.
\item
  Create a research blueprint poster and present your proposed research design/methods to the class for peer review. \textbf{{[}CR1e{]} {[}CR1g{]}} \index{CR1e} \index{CR1g}
\item
  Based on your approved inquiry proposal form and peer review comments on your research blueprint presentation, develop an expanded and refined research proposal, which will serve as a template for your initial paper draft. The research proposal should contain the following sections:

  \begin{itemize}
  \tightlist
  \item
    Research question and hypothesis
  \item
    Literature review
  \item
    Proposed research methods
  \end{itemize}
\item
  Continue weekly PREP check-ins. In this unit, our check-ins will focus on the following items: \textbf{{[}CR4b{]}} \index{CR4b}

  \begin{itemize}
  \tightlist
  \item
    Continue to build more sources into annotated bibliography.
  \item
    Synthesize annotated bibliography into an initial literature review draft for the research proposal. \textbf{{[}CR1d{]}} \index{CR1d}
  \item
    Align research question with literature review and research methods. By the time you begin applying your research methods, your research question should no longer shift to ensure that you are not trying to make the data fit your question.
  \item
    Demonstrate that you are learning enough about your research methods to apply them properly in your own research. Create a separate folder in your PREP to document your learning process on research methods. \textbf{{[}CR1f{]} {[}CR4a{]}} \index{CR1f} \index{CR4a}
  \end{itemize}
\end{enumerate}

\hypertarget{unit-4-academic-paper-drafts-peer-review}{%
\subsection{Unit 4: Academic Paper Drafts \& Peer Review}\label{unit-4-academic-paper-drafts-peer-review}}

\textbf{January/February}: Complete implementation of research methods. Undergo peer review of academic paper drafts. \textbf{{[}CR3{]} {[}CR1e{]}} \index{CR3} \index{CR1e}

\begin{enumerate}
\def\labelenumi{\arabic{enumi}.}
\item
  Finish conducting your research and documenting your results in your PREP. \textbf{{[}CR4a{]}} \index{CR4a}
\item
  Adapt your research proposal into the first draft of your academic paper. Refine the methods section of your paper to reflect findings from your research. Include a new section that analyzes and evaluates your results. Your conclusion should include limitations of the study and directions for future studies. \textbf{{[}CR3{]}} \index{CR3}

  \begin{itemize}
  \tightlist
  \item
    Submit your initial draft for peer review.
  \item
    File the peer review comments from your classmates into your PREP.
  \end{itemize}
\item
  Develop slides on your research method and findings. Present results to the class for peer review. \textbf{{[}CR1e{]} {[}CR1g{]}} \index{CR1e} \index{CR1g}
\end{enumerate}

\hypertarget{unit-5-final-academic-paper-presentation-and-oral-defense}{%
\subsection{Unit 5: Final Academic Paper, Presentation, and Oral Defense}\label{unit-5-final-academic-paper-presentation-and-oral-defense}}

\textbf{March/April}: Complete and submit final academic paper. Conduct 20-minute presentation with oral defense. \textbf{{[}CR3{]} {[}CR5{]}} \index{CR3} \index{CR5}

\begin{enumerate}
\def\labelenumi{\arabic{enumi}.}
\item
  Incorporate peer review feedback into the second draft of the paper. \textbf{{[}CR1e{]}} \index{CR1e}

  \begin{itemize}
  \tightlist
  \item
    Submit your second draft for a final round of peer review.
  \item
    File the peer review comments from your classmates into your PREP.
  \end{itemize}
\item
  Refer to pages 58--59 of the \href{https://apcentral.collegeboard.org/pdf/ap-research-course-and-exam-description.pdf}{AP Research Course and Exam Description} for the list of oral defense questions. You will receive one question per section for a total of three questions and possibly some follow-up questions. \textbf{{[}CR5{]}} \index{CR5}

  \begin{itemize}
  \tightlist
  \item
    In your PREP, outline some responses to these questions as preparation for your oral defense. You will not know ahead of time which questions the panel will ask, so do not try to memorize responses.
  \end{itemize}
\item
  Finalize academic paper and submit it to AP Digital Portfolio. Your teacher will dedicate class time for students to upload their final papers a few days before the official deadline. \textbf{{[}CR3{]}} \index{CR3}
\item
  Sign up for a 20-minute time slot to present and orally defend your research. Prior to the presentations, we will go over the \href{https://secure-media.collegeboard.org/ap/pdf/ap17-sg-research-presentation.pdf}{presentation and oral defense rubric}. The presentations and oral defense will be recorded. \textbf{{[}CR5{]}} \index{CR5}
\end{enumerate}

\hypertarget{unit-6-beyond-ap-research}{%
\subsection{Unit 6: Beyond AP Research}\label{unit-6-beyond-ap-research}}

\textbf{May/June}: Finalize PREP and begin introduction to research tools necessary for research at the undergraduate and graduate levels.

\begin{enumerate}
\def\labelenumi{\arabic{enumi}.}
\item
  Finalize your PREP with the following points in mind: \textbf{{[}CR4a{]}} \index{CR4a}

  \begin{itemize}
  \item
    Include a readme file that documents the contents and purpose of each folder and file in your PREP. Anyone who reads through your readme file should be able to understand how to navigate your PREP without ever having worked with you during the research process.
  \item
    Finalize data documentation that includes metadata (i.e., data about the data) and step-by-step instructions that show how you used the data in your research methods to arrive at your results. Anyone with your PREP should be able to locate the data documentation file from your readme file descriptions and follow your instructions to replicate your results.
  \end{itemize}
\item
  Explore current best practices of reproducible research.

  \begin{itemize}
  \tightlist
  \item
    We will learn basics of literate and data programming with the following software:

    \begin{itemize}
    \tightlist
    \item
      R (using RStudio)
    \item
      LaTeX (using TeXStudio)
    \item
      R Sweave (Rnw files = R + LaTeX)
    \item
      R Markdown (R + Markdown to produce HTML, Word, LaTeX, and PDF outputs)
    \end{itemize}
  \item
    We will explore the basic concepts behind version control using Git and GitHub.

    \begin{itemize}
    \tightlist
    \item
      Instead of saving multiple versions of the same file with version numbers appended to the file name, you can use just one file and commit changes to a repository, which will store metadata about each version of the file.
    \end{itemize}
  \end{itemize}
\item
  Develop a basic static website to showcase your research for college applications and future employment.

  \begin{itemize}
  \tightlist
  \item
    We will use the R \texttt{blogdown} package to develop a basic static website with Hugo, an open-source website generator.
  \item
    Your research website may include the following elements:

    \begin{itemize}
    \tightlist
    \item
      About section
    \item
      Research portfolio
    \item
      Research blog posts
    \item
      Publication section
    \end{itemize}
  \item
    We will also go through the process of hosting the website, but this step is completely optional.
  \end{itemize}
\end{enumerate}

\newpage

\hypertarget{references}{%
\section*{References}\label{references}}
\addcontentsline{toc}{section}{References}

\setlength{\parindent}{-0.2in}
\setlength{\leftskip}{0.2in}
\setlength{\parskip}{8pt}

\noindent

\hypertarget{refs}{}
\leavevmode\hypertarget{ref-cb}{}%
College Board. (2019). \emph{AP Capstone Implementation Guide 2019--20}. \url{https://apcentral.collegeboard.org/pdf/ap-capstone-implementation-guide.pdf}.

\leavevmode\hypertarget{ref-syllabus}{}%
College Board. (n.d.). \emph{Syllabus Development Guide: AP Research}. \url{https://secure-media.collegeboard.org/digitalServices/pdf/ap/ap-course-audit/ap-research-syllabus-development-guide.pdf}.

%manually set hanging indents for references
% \pdfbookmark[0]{\indexname}{Index}
\printindex


\end{document}
